\documentclass{article}
\usepackage{amsmath, amsthm, amssymb, amsfonts}
\usepackage{thmtools}
\usepackage{graphicx}
\usepackage{setspace}
\usepackage{geometry}
\usepackage{float}
\usepackage{hyperref}
\usepackage[utf8]{inputenc}
\usepackage[english]{babel}
\usepackage{framed}
\usepackage[dvipsnames]{xcolor}
\usepackage{tcolorbox}

\colorlet{LightGray}{White!90!Periwinkle}
\colorlet{LightOrange}{Orange!15}
\colorlet{LightGreen}{Green!15}

\newcommand{\HRule}[1]{\rule{\linewidth}{#1}}

\declaretheoremstyle[name=Theorem]{thmsty}
\declaretheorem[style=thmsty,numberwithin=section]{theorem}
\tcolorboxenvironment{theorem}{colback=LightGray}

\declaretheoremstyle[name=Proposition]{prosty}
\declaretheorem[style=prosty,numberlike=theorem]{proposition}
\tcolorboxenvironment{proposition}{colback=LightOrange}

\declaretheoremstyle[name=Principle]{prcpsty}
\declaretheorem[style=prcpsty,numberlike=theorem]{principle}
\tcolorboxenvironment{principle}{colback=LightGreen}

\setstretch{1.2}
\geometry{
    textheight=9in,
    textwidth=5.5in,
    top=1in,
    headheight=12pt,
    headsep=25pt,
    footskip=30pt
}

\title{ \normalsize \textsc{}
		\\ [2.0cm]
		\HRule{1.5pt} \\
		\LARGE \textbf{\uppercase{Linear Algebra Lecture Notes}}
		\HRule{2.0pt} \\ [0.6cm]  \vspace*{10\baselineskip}
		}
\date{}
\author{\textbf{Yoshita Aligina} \\ 
		Rutgers University \\ 
		September 2024}

\begin{document}

\maketitle
\newpage

\tableofcontents
\newpage

\section{Vectors}

\subsection{Vector Example}
\begin{itemize}
    \item Example vector: $32x$
    \item In $H_2$, $X$ moves to the beginning: $H_1, H_2$
    \item $K$ is specified as $2, 4K, 2$ with $D = 15$, and $1$
    \item Query: What will $D$ be?
\end{itemize}

\section{Matrix Dimensions}

\subsection{Matrix Basics}
\begin{itemize}
    \item $n$: Number of rows
    \item $m$: Number of columns
\end{itemize}

\subsection{Examples of Vectors and Matrices}
For $x \in \mathbb{R}$ (real numbers):
\begin{itemize}
    \item For $x = 1, 2, 3, 17$
    \item $n=1$ corresponds to a one-dimensional space
    \item If $n=2$, it results in a two-dimensional plane, including $x_1$ and $x_2$
\end{itemize}

\section{Linear vs Nonlinear Functions}

\subsection{Properties of Linear Functions}
\begin{itemize}
    \item Additivity: $f(x + y) = f(x) + f(y)$
    \item Homogeneity: $f(cx) = c f(x)$
\end{itemize}

Nonlinear functions do not satisfy these properties. If a function fails to satisfy either additivity or homogeneity, it is considered nonlinear.

\subsection{Linear Function Analysis}
- The general rule for linearity: 
\[
f(c \cdot \text{vector}) = c \cdot f(\text{vector})
\]
- Nonlinearity arises if this relationship introduces extra terms like $c^2$.

\subsection{Example Calculation}
For $f(c \cdot \text{vector})$, calculate the result and verify if nonlinearity is introduced.

\section{General Form of Linear Functions}

The general form of a linear function is:
\[
f(x) = a \cdot x + b
\]
If the function is linear, it will not contain any constants or nonlinear terms.

\section{Matrix Multiplication and Dimensions}

\subsection{Matrix Representation}
For a matrix $A$ with dimensions $m \times n$:
\begin{itemize}
    \item $m$: Number of rows
    \item $n$: Number of columns
\end{itemize}

\subsection{Matrix Multiplication}
To multiply matrix $A$ by vector $x$, the dimensions must match:
\[
(A \times x)_i = \sum_{j=1}^{n} a_{ij} x_j
\]
The result will be an $m \times 1$ vector.

\section{Linear Systems of Equations}

\subsection{System Representation}
A system of linear equations can be written as:
\[
A x = b
\]
Where:
\begin{itemize}
    \item $A$: $m \times n$ matrix of coefficients
    \item $x$: $n \times 1$ vector of variables
    \item $b$: $m \times 1$ vector of constants
\end{itemize}

\subsection{Example System}
\[
A = \begin{bmatrix}
a_{11} & a_{12} & \dots & a_{1n} \\
a_{21} & a_{22} & \dots & a_{2n} \\
\vdots & \vdots & \ddots & \vdots \\
a_{m1} & a_{m2} & \dots & a_{mn}
\end{bmatrix}
,
x = \begin{bmatrix}
x_1 \\
x_2 \\
\vdots \\
x_n
\end{bmatrix}
,
b = \begin{bmatrix}
b_1 \\
b_2 \\
\vdots \\
b_m
\end{bmatrix}
\]

\section{Gaussian Elimination}

Gaussian elimination is a method for solving systems of linear equations. It transforms a matrix into row-echelon form using row operations:
\begin{itemize}
    \item Row swaps
    \item Multiplying a row by a constant
    \item Adding or subtracting rows
\end{itemize}

\subsection{Back Substitution}
Once the matrix is in row-echelon form, back substitution can be used to find the solution starting from the last row upwards.

\section{Matrix Multiplication Example}

Given matrix $A$ of size $2 \times 3$ and vector $x$ of size $3 \times 1$:
\[
A = \begin{bmatrix} 1 & 2 & 3 \\ 4 & 5 & 6 \end{bmatrix}
,
x = \begin{bmatrix} x_1 \\ x_2 \\ x_3 \end{bmatrix}
\]
The product $A \times x$ results in:
\[
A \times x = \begin{bmatrix} 1 \cdot x_1 + 2 \cdot x_2 + 3 \cdot x_3 \\ 4 \cdot x_1 + 5 \cdot x_2 + 6 \cdot x_3 \end{bmatrix}
\]

\section{Consistency in Linear Systems}

A system is consistent if there is at least one solution. The stability of the system is determined by how small changes in $b$ affect the solution.

\section{Range and Image of a Function}

A point $y$ is in the range of function $f$ if there exists an $x$ in the domain such that:
\[
f(x) = y
\]

\section{Example of a System with Single Equation}

When $m = 1$ and $n = 1$, the system simplifies to a single equation:
\[
a_{11} x_1 = b_1
\]

\end{document}
