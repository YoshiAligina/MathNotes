\documentclass{article}

\usepackage{amsmath, amsthm, amssymb, amsfonts}
\usepackage{thmtools}
\usepackage{graphicx}
\usepackage{setspace}
\usepackage{geometry}
\usepackage{float}
\usepackage{hyperref}
\usepackage[utf8]{inputenc}
\usepackage[english]{babel}
\usepackage{framed}
\usepackage[dvipsnames]{xcolor}
\usepackage{tcolorbox}

\colorlet{LightGray}{White!90!Periwinkle}
\colorlet{LightOrange}{Orange!15}
\colorlet{LightGreen}{Green!15}

\newcommand{\HRule}[1]{\rule{\linewidth}{#1}}

\declaretheoremstyle[name=Theorem,]{thmsty}
\declaretheorem[style=thmsty,numberwithin=section]{theorem}
\tcolorboxenvironment{theorem}{colback=LightGray}

\declaretheoremstyle[name=Proposition,]{prosty}
\declaretheorem[style=prosty,numberlike=theorem]{proposition}
\tcolorboxenvironment{proposition}{colback=LightOrange}

\declaretheoremstyle[name=Principle,]{prcpsty}
\declaretheorem[style=prcpsty,numberlike=theorem]{principle}
\tcolorboxenvironment{principle}{colback=LightGreen}

\setstretch{1.2}
\geometry{
    textheight=9in,
    textwidth=5.5in,
    top=1in,
    headheight=12pt,
    headsep=25pt,
    footskip=30pt
}

\title{ \normalsize \textsc{}
		\\ [2.0cm]
		\HRule{1.5pt} \\
		\LARGE \textbf{\uppercase{Discrete Mathematics Notes}
		\HRule{2.0pt} \\ [0.6cm] \LARGE{Chapter 1: Logic and Proofs} \vspace*{10\baselineskip}}
		}
\date{}
\author{\textbf{Yoshita Aligina} \\ 
		Rutgers University \\ 
		September 2024}

\begin{document}

\maketitle
\newpage

\tableofcontents
\newpage

% ------------------------------------------------------------------------------

\section{Propositional Logic}

\subsection{Propositions}
\begin{itemize}
    \item A \textbf{proposition} is a declarative sentence that is either true or false, but not both.
    \item Examples of propositions:
    \begin{itemize}
        \item "The sky is blue." (True)
        \item "2 + 2 = 5." (False)
    \end{itemize}
    \item Non-examples of propositions:
    \begin{itemize}
        \item "What time is it?" (Not declarative)
        \item "x + 2 = 4." (Not a proposition, as the truth value depends on $x$)
    \end{itemize}
\end{itemize}

\subsection{Logical Connectives}
\begin{itemize}
    \item The main logical connectives are:
    \begin{itemize}
        \item \textbf{Negation} ($\neg p$): If $p$ is true, then $\neg p$ is false, and vice versa.
        \item \textbf{Conjunction} ($p \land q$): True if both $p$ and $q$ are true, otherwise false.
        \item \textbf{Disjunction} ($p \lor q$): True if at least one of $p$ or $q$ is true, otherwise false.
        \item \textbf{Implication} ($p \to q$): False only when $p$ is true and $q$ is false, otherwise true.
        \item \textbf{Biconditional} ($p \leftrightarrow q$): True if $p$ and $q$ have the same truth value.
    \end{itemize}
\end{itemize}

\subsection{Truth Tables}
\begin{itemize}
    \item Truth tables provide a systematic way to determine the truth value of a proposition based on its components.
    \item Example: Truth table for $p \to q$.
    \begin{center}
        \begin{tabular}{|c|c|c|}
            \hline
            $p$ & $q$ & $p \to q$ \\
            \hline
            T & T & T \\
            T & F & F \\
            F & T & T \\
            F & F & T \\
            \hline
        \end{tabular}
    \end{center}
\end{itemize}

\subsection{Equivalence of Propositions}
\begin{itemize}
    \item Two propositions are \textbf{logically equivalent} if they have the same truth value in all possible cases. We write $p \equiv q$ to denote this.
    \item Common equivalences:
    \begin{itemize}
        \item De Morgan's Laws: $\neg (p \land q) \equiv \neg p \lor \neg q$, $\neg (p \lor q) \equiv \neg p \land \neg q$
        \item Implication: $p \to q \equiv \neg p \lor q$
    \end{itemize}
\end{itemize}

% ------------------------------------------------------------------------------
\section{Predicate Logic}

\subsection{Predicates and Quantifiers}
\begin{itemize}
    \item A \textbf{predicate} is a function $P(x)$ that becomes a proposition when $x$ is given a specific value.
    \item The two main quantifiers are:
    \begin{itemize}
        \item \textbf{Universal Quantifier} ($\forall x \ P(x)$): True if $P(x)$ is true for every $x$ in the domain.
        \item \textbf{Existential Quantifier} ($\exists x \ P(x)$): True if $P(x)$ is true for at least one $x$ in the domain.
    \end{itemize}
\end{itemize}

\subsection{Negating Quantified Statements}
\begin{itemize}
    \item Negating a universally quantified statement: $\neg (\forall x \ P(x)) \equiv \exists x \ \neg P(x)$
    \item Negating an existentially quantified statement: $\neg (\exists x \ P(x)) \equiv \forall x \ \neg P(x)$
\end{itemize}

% ------------------------------------------------------------------------------
\section{Methods of Proof}

\subsection{Direct Proof}
\begin{itemize}
    \item A direct proof assumes the premise $p$ is true and logically deduces that $q$ must also be true.
\end{itemize}

\subsection{Proof by Contraposition}
\begin{itemize}
    \item To prove $p \to q$, prove $\neg q \to \neg p$ instead, as these are logically equivalent.
\end{itemize}

\subsection{Proof by Contradiction}
\begin{itemize}
    \item Assume $p$ is true and $q$ is false, then derive a contradiction.
\end{itemize}

\subsection{Proof by Exhaustion}
\begin{itemize}
    \item Check all possible cases to confirm that the statement holds for all of them.
\end{itemize}

% ------------------------------------------------------------------------------
\section{Common Mistakes in Proofs}

\begin{itemize}
    \item Assuming what is to be proven (begging the question).
    \item Confusing a statement and its converse.
    \item Misusing quantifiers (e.g., confusing $\forall x$ with $\exists x$).
\end{itemize}



% ------------------------------------------------------------------------------
\bibliographystyle{IEEEtran}
\bibliography{References.bib}

% ------------------------------------------------------------------------------
\end{document}
