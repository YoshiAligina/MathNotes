
\documentclass{article}
\usepackage{amsmath, amsthm, amssymb, amsfonts}
\usepackage{thmtools}
\usepackage{graphicx}
\usepackage{setspace}
\usepackage{geometry}
\usepackage{float}
\usepackage{hyperref}
\usepackage[utf8]{inputenc}
\usepackage[english]{babel}
\usepackage{framed}
\usepackage[dvipsnames]{xcolor}
\usepackage{tcolorbox}

\colorlet{LightGray}{White!90!Periwinkle}
\colorlet{LightOrange}{Orange!15}
\colorlet{LightGreen}{Green!15}

\newcommand{\HRule}[1]{\rule{\linewidth}{#1}}

\declaretheoremstyle[name=Theorem]{thmsty}
\declaretheorem[style=thmsty,numberwithin=section]{theorem}
\tcolorboxenvironment{theorem}{colback=LightGray}

\declaretheoremstyle[name=Proposition]{prosty}
\declaretheorem[style=prosty,numberlike=theorem]{proposition}
\tcolorboxenvironment{proposition}{colback=LightOrange}

\declaretheoremstyle[name=Principle]{prcpsty}
\declaretheorem[style=prcpsty,numberlike=theorem]{principle}
\tcolorboxenvironment{principle}{colback=LightGreen}

\setstretch{1.2}
\geometry{
    textheight=9in,
    textwidth=5.5in,
    top=1in,
    headheight=12pt,
    headsep=25pt,
    footskip=30pt
}

\title{
    \normalsize \textsc{}
    \\ [2.0cm]
    \HRule{1.5pt} \\
    \LARGE \textbf{\uppercase{Discrete Mathematics Notes}}
    \HRule{2.0pt} \\
    [0.6cm]
    \LARGE{Comprehensive Guide}
    \vspace*{10\baselineskip}
}

\date{}
\author{\textbf{Your Name} \\ University Name \\ September 2024}

\begin{document}

\maketitle
\newpage
\tableofcontents
\newpage

\section{Introduction to Discrete Mathematics}

\subsection{Definition and Scope}
Discrete mathematics is the study of mathematical structures that are fundamentally discrete rather than continuous. It deals with objects that can assume only distinct, separated values.

\subsection{Importance in Computer Science}
Discrete mathematics forms the foundation of computer science, providing the mathematical tools necessary for:
\begin{itemize}
    \item Algorithm analysis
    \item Data structure design
    \item Cryptography
    \item Computer architecture
    \item Software engineering
\end{itemize}

\section{Set Theory}

\subsection{Basic Concepts}
A set is a well-defined collection of distinct objects.

\subsubsection{Set Notation}
Sets are typically denoted using curly braces:
$$A = \{1, 2, 3, 4, 5\}$$

\subsubsection{Set Builder Notation}
$$B = \{x \mid x \text{ is an even number less than 10}\}$$

\subsection{Set Operations}

\subsubsection{Union}
The union of sets A and B, denoted $A \cup B$, contains all elements that are in A, or B, or both.
$$A \cup B = \{x \mid x \in A \text{ or } x \in B\}$$

\subsubsection{Intersection}
The intersection of sets A and B, denoted $A \cap B$, contains all elements that are in both A and B.
$$A \cap B = \{x \mid x \in A \text{ and } x \in B\}$$

\subsubsection{Difference}
The difference of sets A and B, denoted $A \setminus B$, contains all elements that are in A but not in B.
$$A \setminus B = \{x \mid x \in A \text{ and } x \notin B\}$$

\subsubsection{Complement}
The complement of set A, denoted $A^c$, contains all elements in the universal set that are not in A.
$$A^c = \{x \mid x \notin A\}$$

\subsection{Set Properties}

\begin{theorem}[Distributive Laws]
For any sets A, B, and C:
\begin{align*}
A \cup (B \cap C) &= (A \cup B) \cap (A \cup C) \\
A \cap (B \cup C) &= (A \cap B) \cup (A \cap C)
\end{align*}
\end{theorem}

\begin{theorem}[De Morgan's Laws]
For any sets A and B:
\begin{align*}
(A \cup B)^c &= A^c \cap B^c \\
(A \cap B)^c &= A^c \cup B^c
\end{align*}
\end{theorem}

\section{Logic and Proofs}

\subsection{Propositional Logic}

\subsubsection{Propositions}
A proposition is a declarative statement that is either true or false, but not both.

\subsubsection{Logical Operators}

\paragraph{Negation ($\neg$)}
The negation of a proposition p, denoted $\neg p$, is true when p is false and false when p is true.

\paragraph{Conjunction ($\land$)}
The conjunction of propositions p and q, denoted $p \land q$, is true only when both p and q are true.

\paragraph{Disjunction ($\lor$)}
The disjunction of propositions p and q, denoted $p \lor q$, is true when at least one of p or q is true.

\paragraph{Exclusive OR ($\oplus$)}
The exclusive OR of propositions p and q, denoted $p \oplus q$, is true when exactly one of p or q is true.

\paragraph{Implication ($\rightarrow$)}
The implication $p \rightarrow q$ is false only when p is true and q is false.

\paragraph{Biconditional ($\leftrightarrow$)}
The biconditional $p \leftrightarrow q$ is true when p and q have the same truth value.

\subsection{Predicate Logic}

\subsubsection{Predicates}
A predicate is a proposition whose truth depends on the value of one or more variables.

\subsubsection{Quantifiers}

\paragraph{Universal Quantifier ($\forall$)}
$\forall x P(x)$ means "for all x, P(x) is true"

\paragraph{Existential Quantifier ($\exists$)}
$\exists x P(x)$ means "there exists an x such that P(x) is true"

\subsection{Methods of Proof}

\subsubsection{Direct Proof}
Assume the hypothesis and use logical reasoning to reach the conclusion.

\subsubsection{Proof by Contraposition}
To prove $p \rightarrow q$, prove its contrapositive $\neg q \rightarrow \neg p$.

\subsubsection{Proof by Contradiction}
Assume the negation of the statement and derive a contradiction.

\subsubsection{Mathematical Induction}
\begin{enumerate}
    \item Prove the base case (usually n = 1 or n = 0).
    \item Assume the statement is true for some k.
    \item Prove the statement is true for k + 1.
\end{enumerate}

\section{Number Theory}

\subsection{Divisibility}
An integer a divides an integer b if there exists an integer k such that b = ak.

\subsection{Prime Numbers}
A prime number is a natural number greater than 1 that has no positive divisors other than 1 and itself.

\subsection{Greatest Common Divisor (GCD)}
The GCD of two or more integers is the largest positive integer that divides each of the integers.

\begin{theorem}[Euclidean Algorithm]
To find the GCD of a and b:
\begin{enumerate}
    \item If b = 0, return a.
    \item Otherwise, return GCD(b, a mod b).
\end{enumerate}
\end{theorem}

\subsection{Least Common Multiple (LCM)}
The LCM of two or more integers is the smallest positive integer that is divisible by each of the integers.

\begin{theorem}
For any two positive integers a and b:
$$LCM(a,b) = \frac{|ab|}{GCD(a,b)}$$
\end{theorem}

\section{Functions}

\subsection{Definition}
A function f from a set A to a set B is a rule that assigns to each element of A exactly one element of B.

\subsection{Types of Functions}

\subsubsection{Injective (One-to-One)}
A function f is injective if $f(a) = f(b)$ implies $a = b$ for all a and b in the domain.

\subsubsection{Surjective (Onto)}
A function f from A to B is surjective if for every element b in B, there exists an element a in A such that f(a) = b.

\subsubsection{Bijective}
A function is bijective if it is both injective and surjective.

\subsection{Composition of Functions}
If f : A → B and g : B → C, then the composition g ∘ f : A → C is defined as (g ∘ f)(x) = g(f(x)).

\section{Graph Theory}

\subsection{Basic Definitions}
A graph G = (V, E) consists of a set V of vertices and a set E of edges connecting pairs of vertices.

\subsection{Types of Graphs}

\subsubsection{Undirected Graphs}
Edges have no direction.

\subsubsection{Directed Graphs (Digraphs)}
Edges have a direction.

\subsubsection{Weighted Graphs}
Edges have associated weights or costs.

\subsection{Graph Properties}

\subsubsection{Connectivity}
A graph is connected if there is a path between every pair of vertices.

\subsubsection{Eulerian Paths and Circuits}
An Eulerian path visits every edge exactly once. An Eulerian circuit is an Eulerian path that starts and ends at the same vertex.

\subsubsection{Hamiltonian Paths and Cycles}
A Hamiltonian path visits every vertex exactly once. A Hamiltonian cycle is a Hamiltonian path that returns to the starting vertex.

\subsection{Graph Algorithms}

\subsubsection{Depth-First Search (DFS)}
Explores as far as possible along each branch before backtracking.

\subsubsection{Breadth-First Search (BFS)}
Explores all the neighbor nodes at the present depth prior to moving on to the nodes at the next depth level.

\subsubsection{Dijkstra's Algorithm}
Finds the shortest path between nodes in a graph.

\section{Combinatorics}

\subsection{Counting Principles}

\subsubsection{Sum Rule}
If a task can be done in m ways, and another task can be done in n ways, and the two tasks cannot be done at the same time, then there are m + n ways to do either task.

\subsubsection{Product Rule}
If a task can be done in m ways, and another independent task can be done in n ways, then the two tasks can be done together in m × n ways.

\subsection{Permutations}
A permutation of n distinct objects is an ordered arrangement of these objects.
$$P(n) = n!$$

\subsection{Combinations}
A combination of n objects taken r at a time is a selection of r objects from a set of n objects where order doesn't matter.
$$C(n,r) = \binom{n}{r} = \frac{n!}{r!(n-r)!}$$

\subsection{Binomial Theorem}
$$(x+y)^n = \sum_{k=0}^n \binom{n}{k} x^{n-k} y^k$$

\section{Probability Theory}

\subsection{Basic Probability}
The probability of an event A is the number of favorable outcomes divided by the total number of possible outcomes, assuming all outcomes are equally likely.

\subsection{Conditional Probability}
The probability of event A given that event B has occurred is:
$$P(A|B) = \frac{P(A \cap B)}{P(B)}$$

\subsection{Bayes' Theorem}
$$P(A|B) = \frac{P(B|A)P(A)}{P(B)}$$

\subsection{Random Variables}
A random variable is a variable whose possible values are numerical outcomes of a random phenomenon.

\subsubsection{Expected Value}
For a discrete random variable X:
$$E(X) = \sum_{x} x P(X = x)$$

\subsubsection{Variance}
$$Var(X) = E((X - E(X))^2) = E(X^2) - (E(X))^2$$

\section{Recurrence Relations}

\subsection{Definition}
A recurrence relation is an equation that recursively defines a sequence where each term is a function of the previous terms.

\subsection{Linear Recurrence Relations}
A linear recurrence relation of degree k has the form:
$$a_n = c_1a_{n-1} + c_2a_{n-2} + ... + c_ka_{n-k} + f(n)$$

\subsection{Solving Recurrence Relations}

\subsubsection{Characteristic Equation Method}
For homogeneous linear recurrence relations.

\subsubsection{Method of Generating Functions}
Useful for both homogeneous and non-homogeneous recurrence relations.

\section{Formal Languages and Automata Theory}

\subsection{Formal Languages}
A formal language is a set of strings of symbols that are constructed according to specific rules.

\subsection{Regular Expressions}
A notation for describing patterns in strings.

\subsection{Finite Automata}
A mathematical model of computation that can be in one of a finite number of states.

\subsubsection{Deterministic Finite Automata (DFA)}
A finite state machine that accepts or rejects strings of symbols and only produces a unique computation of the automaton for each input string.

\subsubsection{Non-deterministic Finite Automata (NFA)}
Similar to a DFA, but allows for multiple possible transitions from a given state on the same input.

\subsection{Context-Free Grammars}
A set of recursive rewriting rules used to generate patterns of strings.

\section{Cryptography}

\subsection{Symmetric Key Cryptography}
Uses the same key for both encryption and decryption.

\subsection{Public Key Cryptography}
Uses a public key for encryption and a private key for decryption.

\subsection{RSA Algorithm}
A widely used public-key cryptosystem based on the difficulty of factoring large integers.

\section{Boolean Algebra}

\subsection{Boolean Functions}
Functions that operate on binary inputs and produce binary outputs.

\subsection{Logic Gates}
Physical implementations of Boolean functions.

\subsection{Minimization of Boolean Functions}
Techniques to simplify Boolean expressions, such as Karnaugh maps and the Quine-McCluskey algorithm.

\section{Summary}

Discrete mathematics provides the foundation for many areas of computer science and mathematics. Key topics include:

\begin{itemize}
    \item Set theory and logic
    \item Number theory and functions
    \item Graph theory and combinatorics
    \item Probability and recurrence relations
    \item Formal languages and automata theory
    \item Cryptography and Boolean algebra
\end{itemize}

Understanding these concepts

Citations:
[1] https://ppl-ai-file-upload.s3.amazonaws.com/web/direct-files/31605656/a37cd280-d4a0-4ccf-89ad-a9216a58806a/regressionvro.txt
[2] https://ppl-ai-file-upload.s3.amazonaws.com/web/direct-files/31605656/23952b69-fd2d-4cef-9aaa-739cd50c6d65/paste.txt
[3] https://ppl-ai-file-upload.s3.amazonaws.com/web/direct-files/31605656/8a019c2b-a96b-4442-8d21-3a9e57ecbc31/paste-2.txt